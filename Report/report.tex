\documentclass[a4paper]{article}

\usepackage{amsmath}
\usepackage{amsfonts}
\usepackage{amssymb}
\usepackage{graphicx}
\usepackage{xcolor}
\usepackage{soul}
\usepackage{geometry}
\usepackage{titling}
\usepackage{csquotes}
\geometry{
%    left=20mm,   
%    right=20mm,
%    top=20mm,
    bottom=20mm,
 }
\setlength{\droptitle}{-27mm}
\definecolor{grey}{rgb}{0.827, 0.827, 0.827}
\sethlcolor{grey}

% Bibliography 
\usepackage[backend=biber, style=apa, citestyle=authoryear, sorting=nyt]{biblatex}
\addbibresource{References.bib}

% Define a small matrix
\newenvironment{psmallmatrix}
  {\left(\begin{smallmatrix}}
  {\end{smallmatrix}\right)}

% Define regex function names
\def \variable {search}
\newcommand \search {\textcolor{teal}{\textbf{re.search(\emph{pattern}, \emph{string}, \emph{flags=re.I})}}}

\def \variable {finditer}
\newcommand \finditer {\textcolor{teal}{\textbf{re.finditer(\emph{pattern}, \emph{string}, \emph{flags=re.I})}}}

% Define Regex Match objects
\def \variable {matchStart}
\newcommand \matchStart {\textcolor{teal}{\textbf{match.start()}}}

\def \variable {matchEnd}
\newcommand \matchEnd {\textcolor{teal}{\textbf{match.end()}}}

% Define fertliser function
\def \variable {fertiliser}
\newcommand \fertiliser {\textcolor{teal}{fertiliser(an, ap, bn, bp, n, p)}}


\author{Shridhar Thorat: n10817239}
\title{CAB203 Problem solving}

\begin{document}
\maketitle
\date{}

\section{Regular languages and finite state automata}
In order to censor the articles: '\emph{a}', '\emph{an}' and  '\emph{the}', and append a student's number in their emails, a Python function \textcolor{teal}{\textbf{censor(s)}} was developed. This function applies the required censoring to a string \textcolor{teal}{\textbf{s}}, replacing each letter in an article with the pound sign: \textcolor{teal}{\textbf{\#}}, and appends a student number (including a single whitespace) if the string was censored.
\vspace{5mm}

\noindent
\emph{Syntax and patterns:}
\vspace{2mm}

\noindent
This function took advantage of the Python package \textcolor{teal}{\textbf{re}}, which provides regular expression syntax and functions. The syntax used were; \textcolor{teal}{\textbackslash b} and \textcolor{teal}{( )}. The syntax \textcolor{teal}{\textbackslash b} defines the beginning and ending of a word boundary, while \textcolor{teal}{( )} defines a capture group.

The regex patterns for the three articles in Python were chosen as below. Where \emph{r} simply defines a raw string.
\begin{center}
    \begin{minipage}[c]{1\linewidth}
        \begin{itemize}
            \item \emph{r}\enquote{\textbackslash b(a)\textbackslash b} only matches the word \enquote{\emph{a}}
            \item \emph{r}\enquote{\textbackslash b(an)\textbackslash b} only matches the word \enquote{\emph{an}}
            \item \emph{r}\enquote{\textbackslash b(the)\textbackslash b} only matches the word \enquote{\emph{the}}
        \end{itemize} 
    \end{minipage}
\end{center}
These regex patterns are checked in the string \textcolor{teal}{s} from left to right. The patterns can be interpreted as below.

\begin{itemize}
    \item The first \hl{\textbackslash b} sets the starting word boundary \autocite{a2022_re}.
    \item The article in \hl{()} defines the character capture group. Where only characters in this group are matches \autocite{a2022_re}.
    \item The last \hl{\textbackslash b} sets the end of the word boundary \autocite{a2022_re}.
\end{itemize}
Thus the patterns only match words in the capture group. However, the articles must be matched regardless of capitalisation - which isn't done by the pattern. Instead of modifying the pattern, the flag \textcolor{teal}{re.IGNORECASE} is used.
\vspace{5mm}


\noindent
\emph{Flags used:}
\vspace{2mm}

\noindent
The flag \textcolor{teal}{re.IGNORECASE} (\textcolor{teal}{re.I}) is used as a function parameter. As the name suggests, it performs case-\emph{insensitive} matching for a pattern \autocite{a2022_re}. For instance, the pattern \emph{r}\enquote{\textbackslash b(the)\textbackslash b} would match \enquote{\emph{The}}, \enquote{\emph{ThE}}, etc, when the flag is used in any function.

Now we require functions which can determine if a match (article) exists in a string and all the matches for each of the patterns.
\vspace{5mm}

\noindent
\emph{Functions used:}
\vspace{2mm}

\noindent
In order to find the first location where a pattern produces a match, the regular function \search \ can be used. As the description, this function is used to determine whether any of the patterns exist, regardless of capitalisation (as defined by the flag) \autocite{a2022_re}. If the patterns exists, it can then be censored. However, this requires determining their location within the string.
\vspace{2mm}

\noindent
This can be done by using the function \finditer which scans the \textcolor{teal}{string} left-to-right and returns an iterator that yields \emph{match} objects which match the regex \textcolor{teal}{pattern} \autocite{a2022_re}.

I.e, it returns every match to the pattern, as a regex \emph{match} object. These objects have 2 attributes useful in this problem; the \matchStart \ and \matchEnd. These gives the starting and ending index of a match respectively \autocite{a2022_re}. Given the start and end indices, the string can be censored with a pound \# between these indices - thus censoring the article. Applying this for each match (i.e. article), the whole string can be censored.


\section{Linear algebra}
Before implementing the problem in Python, it needs to be defined using mathematical terminology.
\vspace{2mm}

\noindent
From the premise, 2 simultaneous equations that solve for nitrogen \textbf{\textcolor{teal}{n}}, and phosphate \textbf{\textcolor{teal}{p}} can be created.
\begin{center} $n=0.3A + 0.1B$ \end{center}
\begin{center} $p=0.2A + 0.1B$ \end{center}
\begin{center} Where $A$ and $B$ represent 1 kilogram of fertiliser $A$ and $B$ respectively. \end{center}

\noindent
If we consider that the ratios of nitrogen and phosphate in the two fertilisers are unknown, where
\begin{itemize}
    \item \textbf{\textcolor{teal}{an}} and \textbf{\textcolor{teal}{bn}} are the amounts of nitrogen in 1kg of fertiliser A and B respectively
    \item \textbf{\textcolor{teal}{ap}} and \textbf{\textcolor{teal}{bp}} are the amounts of phosphate in 1kg of fertiliser A and B respectively
\end{itemize}
Then the equations can be re-written As
\begin{center} $n=anA + bnB$ \end{center}
\begin{center} $p=apA + bpB$ \end{center}

\noindent
In matrix form:
\begin{equation*}
    \begin{bmatrix} n \\p \end{bmatrix} =
    \begin{bmatrix} an & bn \\ ap & bp \end{bmatrix}
    \begin{bmatrix} A \\B \end{bmatrix}
\end{equation*}
\begin{center} or $\mathbf{r} = C\mathbf{f}$ \end{center}

\noindent
Since the amount of fertiliser A and B is unknown, we can rearrange the equation as such
\begin{equation}
    \begin{bmatrix} A \\B \end{bmatrix} =
    \begin{bmatrix} an & bn \\ ap & bp \end{bmatrix}^{-1}
    \begin{bmatrix} n \\p \end{bmatrix}
\end{equation}
\begin{equation} or \ \ \mathbf{f} = C^{-1}\mathbf{r} \end{equation}

\noindent
The inverse of $C$ can be written using the rule for inverses of 2-by-2 matrices:
\begin{equation*}
    C^{-1} = 
    \begin{bmatrix} an & bn \\ ap & bp \end{bmatrix}^{-1} =
    det(C)\cdot \begin{bmatrix} bp & -bn \\ -ap & an \end{bmatrix}
\end{equation*}
Where
\begin{equation*}
    det(C) = \frac{1}{an\cdot bp - bn\cdot ap} 
\end{equation*}

\noindent
From the formula for the inver of $C$, it can be observed that the inverse will not exist if the determinant ($det(C)$) is $0$. And thus by equation $(2)$, matrix (technically vector) $\mathbf{f}$ does not have a unique solution. 
\vspace{2mm}

\noindent
This problem was implemented in python using the function \fertiliser, which returns a tuple \textcolor{teal}{(a,b)} which represents the amount of fertiliser $A$ and $B$. Using the Python package \textbf{numpy}, the function returns \textcolor{teal}{None} when the determinant of matrix $C$ is $0$. Additionally, sicne the amount of fertiliser cannot be negative, it also returns \textcolor{teal}{None} if \textcolor{teal}{a} or \textcolor{teal}{b} in \textcolor{teal}{(a,b)} are negative.

\printbibliography

\end{document}